\chapter{O problema de caching com predições} 

\section{Framework}
ROBUSTEZ vs COMPETITIVIDADE

CARACTERISTICAS PRATICAS
\subsection{Características}


\section{Oráclo cego}

\begin{itemize}
    \item \textbf{Victor}: Vale descrever o oráculo e os resultados. Quase nenhuma prova.
\end{itemize}

\section{Predictive Marker e Marker Melhorado}

\begin{itemize}
    \item \textbf{Victor}: Descreve o algoritmo do 
    Lykouris e o do Rohatigi. Legal ter mas não essencial: Depois prova do Marker melhorado. BONUS: Deixa a prova predictive marker por último. 
\end{itemize}







\section{Combinando algoritmos como caixa preta}

\begin{itemize}
    \item \textbf{Victor}: Primeiro descrever de forma bem sucinta, talvez nem em uma sessão separada. Faz a prova da caixa preta DEPOIS que tiver não resto. 
\end{itemize}

\section{Cota inferior}


\begin{itemize}
    \item \textbf{Victor}: Enunciar o resultado e discutir sobre a cota inferior (por que o exemplo que ele dá é uma cota inferior para o problema, envolve discussão e desenho que tivemos na lousa).
\end{itemize}

